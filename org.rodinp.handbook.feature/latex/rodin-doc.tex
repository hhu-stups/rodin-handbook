\documentclass{book}
\usepackage[left=2.5cm,top=3cm,right=2.5cm,bottom=3cm]{geometry}
\usepackage[colorlinks=true]{hyperref}
\usepackage{graphicx}
\usepackage{bsymb}
\usepackage{b2latex}
\usepackage{fancyhdr,lastpage,color}
\usepackage{verbatim}
\usepackage{wrapfig}
\usepackage{makeidx}

% Rodin Handbook Version
\newcommand{\versionnr}{2.8}

% Rodin Handbook Version Path. "current" is the newest version of the handbook. This should be changed if we want to build a handbook for another (i.e. older version)
\newcommand{\versionpath}{current}

% Absolute path to handbook
\newcommand{\handbookpath}{http://handbook.event-b.org}

% We generate an index
\makeindex


% defining a if(plastex) environment
\newif\ifplastex
\plastexfalse

\def\doculist#1#2{
\begin{quote}
\hspace{-10mm}
\textrm{\includegraphics[width=7mm]{#2}} % Hack!  We "mark" the image with textrm so that we can use a different CSS-Style in plastex.
\vspace{-8mm}

#1
\end{quote}
}

\def\tick#1{\doculist{#1}{img/tick_64.png}}
\def\info#1{\doculist{#1}{img/info_64.png}}
\def\warning#1{\doculist{#1}{img/warning_64.png}}
\def\pencil#1{\doculist{#1}{img/pencil_64.png}}

% different method to write an ASCII backslash for plastex and normal pdflatex
\ifplastex
  \newcommand{\mybackslash}{\textbackslash}
\else
  % we do not use textbackslash for latex, because it does not use the current font setting
  \newcommand{\mybackslash}{\symbol{`\\}}
\fi

% Path to resources like zip's with machines
% We use a relative path in the html + eclipe version (in order to work offline)
% and an absolute path in the pdf version
\ifplastex
	\newcommand{\filepath}{files/}
\else
	\newcommand{\filepath}{\handbookpath/\versionpath/files/}
\fi

% Use this definition to create a link to the file. The definition takes to arguments. 
% The first argument (1) defines the file name i.e. Celebrity.zip or in case if you saved 
% the file in a subdirectory subdirecotry/Celebrity.zip. The second argument (2) defines 
% the name which should be displayed in the document, i.e. Celebrity Problem Example Download
\def\file#1#2{
\href{\filepath#1}{#2}
}


\title{User Manual for Rodin v.\versionnr}
\author{
Work in Progress\\
Handbook $ $Rev$ $ \\
\\
\href{mailto:rodin-handbook@formalmind.com}{rodin-handbook@formalmind.com}\\
\href{http://handbook.event-b.org}{handbook.event-b.org}
}

\begin{document}        

\maketitle

\ifplastex
\else
\tableofcontents
\fi

\chapter{Introduction}

The Rodin documentation needs improvement.  This has been established at the Deploy Exec Meeting in 2010.  We've been collecting background information and the requirements in the \href{http://wiki.event-b.org/index.php/User_Documentation_Overhaul}{Rodin Wiki}.

What you see here is a working draft of the documentation.  We are grateful for any feedback that you may have (\ref{feedback}).

\section{Overview}

This handbook consists of three parts:

\begin{description}
	\item[Introduction] You are reading the introduction right now.  It helps you to orient yourself and to find information quickly.
	\item[Tutorial] If you are completely new to Rodin, the Tutorial is a good way to get up to speed quickly.  It guides you through installation and usage of the tool and gives you an overview of the Event-B modeling notation.
	\item[Reference] The reference section provides comprehensive documentation of Rodin, and its components.
	\item[Frequently Asked Questions] Common issues are listed by category in the FAQ.
	\item[Index] Particularly for the print version of this handbook, we included an index.  In the electronic versions, you may want to try the search functionality as well.
\end{description}

\subsection{Formats of this Handbook}

The handbook comes in various formats:

\begin{description}
	\item[Eclipse Help] Rodin is shipped with Rodin and can be accessed through the help system.  The handbook will be updated through the standard Rodin update mechanism.
	\item[Online Help] You can access the handbook online at \url{http://handbook.event-b.org}.
	\item[PDF Help] Both online versions also include a link to the PDF version of the handbook.
	\item[Phyiscal Book] If there is enough interest, we may make the handbook available through a print-on-demand retailer.
\end{description}

\subsection{Rodin Wiki}
\label{rodin_wiki}

This handbook is complemented by the Rodin wiki (\url{http://wiki.event-b.org/}).  Sometimes, the handbook will refer to the wiki for more information.  Also, plugin and developer information is usually located in the wiki.

\subsection{Feedback}
\label{feedback}

All online versions of the handbook contain a button or link for feedback.  Work on the handbook continues at least until January 2012, so your feedback will be read and will help to improve this handbook.  You can also submit feedback via email to \texttt{rodin-handbook@formalmind.com}.

\section{Foreword}

It would be nice to recruit somebody for the foreword - maybe Cliff and/or Jean-Raymond.

\section{Conventions}
\label{conventions}

We use the following conventions in this manual:

\tick{Checklists and Milestones are designated with tick. Here we summarize what we want to learn or should have learned so far.}
\info{Useful information and tricks are designated by the information sign.}
\warning{Potential problems and warnings are designated by a warning sign.}
\pencil{Examples and Code are designated by a pencil.}

We use \texttt{typewriter} font for file names and directories.

We use \textsf{sans serif font} for GUI elements like menus and buttons.  Menu actions are depicted by a chain of elements, separated by ``$\rangle$'', e.g. \textsf{File $\rangle$ New $\rangle$ Event-B Component}.

\section{Acknowledgements}
\label{sec:acknowledgements}

The content of this handbook has grown over many years, since the formation of the European Union IST Project RODIN in 2004.  Giving credit to every contributor is almost impossible, attempting to do so would almost certainly omit some people, which is not in the spirit of this work.  It should suffice to say that we extend our gratitude to all contributors to the Rodin Wiki (\ref{rodin_wiki}). In particular, we would like to thank Systerel\footnote{\url{http://www.systerel.fr}} for their significant contributions to the handbook, as they have been the main driver behind the tool and its documentation.

We would also like to thank Cliff Jones, who never gave up the quest for improved Rodin documentation.

The icons that you find throughout this handbook were created by Pixel-Mixer\footnote{\url{http://pixel-mixer.com/}}, who provides them freely.  Thanks!

\section{DEPLOY}
\label{deploy}

This work has been sponsored by the DEPLOY project\footnote{\url{http://www.deploy-project.eu/}}.  DEPLOY is an European Commission Information and Communication Technologies FP7 project.

The overall aim of the EC Information and Communication Technologies FP7 DEPLOY Project is to make major advances in engineering methods for dependable systems through the deployment of formal engineering methods. Formal engineering methods enable greater mastery of complexity than found in traditional software engineering processes. It is the central role played by mechanically-analysed formal models throughout the system development flow that enables mastery of complexity.

As well as leading to big improvements in system dependability, greater mastery of complexity also leads to greater productivity by reducing the expensive test-debug-rework cycle and by facilitating increased reuse of software.

The work of the project is driven by the tasks of achieving and evaluating industrial take-up, initially by DEPLOY's industrial partners, of DEPLOY's methods and tools, together with the necessary further research on methods and tools. 

\section{Creative Commons Legal Code}
\label{sec:cc}        

The work presented here is the result of a collaborative effort
that took many years.  To ensure that access to this work stays free
and to avoid any legal ambiguities, we have decided to formally license
it under the Creative Commons Share-Alike License.

This work, except the cover image, is licensed under the Creative Commons Attribution-ShareAlike 3.0 Unported License. To view a copy of this license, visit \href{http://creativecommons.org/licenses/by-sa/3.0/}{creative\-com\-mons.org/\-licenses/\-by-sa/3.0/} or send a letter to Creative Commons, 444 Castro Street, Suite 900, Mountain View, California, 94041, USA.

The cover image of a Rodin statue was created by Miikka Skaffari (\href{http://www.skaffari.fi/}{www.skaffari.fi}).  It is licensed under a Creative Commons Attribution-NonCommercial 3.0 Unported License.  To view a copy of this license, visit \href{http://creativecommons.org/licenses/by-sa/3.0/}{creative\-com\-mons.org/\-licenses/\-by-sa/3.0/} or send a letter to Creative Commons, 444 Castro Street, Suite 900, Mountain View, California, 94041, USA.




\section{Style Guide}

\info{For now, we will manage the style guide as \LaTeX~together with the rest of the documentation.  We may take it out upon publication.}

\subsubsection{General Stylistic Guidelines}

\begin{itemize}
	\item The Conventions (\ref{conventions}) are part of the style guide.

	\item Use the ``we'' form.

	\item We use British English.
\end{itemize}

\subsubsection{Files}

Files should be saved in the \texttt{files} subdirectory. You can also create a subdirectory. Then, use the definition 

\begin{verbatim} \filepath{1}{2} \end{verbatim} 

to create a link to the file. The definition takes to arguments. The first argument (1) defines the file name i.e. \texttt{Celebrity.zip} or in case if you saved the file in a subdirectory \texttt{subdirecotry/Celebrity.zip}. The second argument (2) defines the name which should be displayed in the document, i.e. ``Celebrity Problem Example Download''.

\warning{Please note, that you only enter the file name without a path before (expected subdirectories). The build script assigns automatically the correct path to the file on the server.}

Here is an example using the definition: \file{Celebrity.zip}{Celebrity Problem Example Download}.

\subsubsection{Images}
\begin{itemize}
	\item Images must be no more than 700 pixels in width (for HTML version)  This is fairly easy for bitmaps (screenshot), pay attention to this regarding how plasTeX converts vector images. (see Latex section below on how to include images)

    \item Screenshots should look neat and consistent.  Horizontal real estate will always be an issue, so please resize the windows before taking the screenshot to keep things readable at 700 pixel width.  (see Latex section below on how to include images)

	\item We use icons from Pixel-Mixer, which are free as long as credit is given: \url{http://www.softicons.com/free-icons/toolbar-icons/basic-icons-by-pixelmixer}

  \item We include Window decoration only when it is really necessary.  If we discuss only some views, we crop the rest away.  Please crop neatly, following edges. If you need a screenshot with window decoration neverthlees we should use always the same Window decoration (i.e. linux ubuntu default decoration style). If you need such a screenshot, please contact Lukas.

  \item Image file names should be all in lower case and not include umlaute or special characters. Use ``\_'' for blanks.

  \item Image files should keep the following rules:
	
\begin{itemize}
		\item Tutorial images should be saved in the sub folder \texttt{img/tutorial} with the prefix ``\texttt{tut\_}'' following the section number. For instance, \texttt{tut\_01\_image1.png}.

		\item FAQ images should be saved in the sub folder \texttt{img/faq} with the prefix ``\texttt{faq\_}''. For instance, \texttt{faq\_image1.png}.

	\end{itemize} 
	\end{itemize} 
\subsubsection{Avoiding Redundancy}

We will reduce (or avoid) redundancy through heavy linking, following these guidelines:

\begin{itemize}
	\item If in doubt, provide the bulk of the information in the Reference section.  For instance, the FAQ entry ``What is Event-B?''  Should simply refer to the Event-B entry in the Reference section.
	\item Web Links should not appear multiple times
	\item Realize web links as footnotes in the Tutorial and FAQ.
	\item Realize web links in a ``See also'' Section in the Reference.
\end{itemize}

\subsubsection{Sections}

\begin{itemize}
	\item We use uppercase for chapter, section, if it refers to a specific one, e.g. ``in Chapter~3''.
	\item Even to subsections we refer to as Section, e.g. ``see Section~2.5''
	\item We have a small number of well-defined chapters, the top level structuring element.
	\item Sections and subsections are numbered.  In the HTML-Versions, they are broken into subpages.
    \item Subsubsections do not receive numbers and are not broken into subpages in the HTML.  Keep this in mind regarding both the reading flow and page sizes.
	\item Avoid linking (ref) to subsubsections, as they don't have a number.  Latex will instead provide a link to the next higher element.  It works, but could create confusion.
	\item Generally, we should avoid gaps in the hierarchy (i.e. having a subsubsection in a section without a subsection in between).\footnote{Coincidentally, this style guide violates this rule. Reason: We want the style guide not broken into subpages, but the proper hierarchy is a section.}
	\item Section labels should be all in lower case. Use ``\_'' for blanks.
	\item We use the prefix ``\texttt{int\_}'' for introduction section labels, ``\texttt{tut\_}'' for tutorial section labels following the section number (i.e. \texttt{tut\_01}) and ``\texttt{faq\_}'' for faq section labels following a short version of the title (i.e. \texttt{faq\_diff\_eventb\_b}). Reference section labels have no prefix.
\end{itemize}

\subsubsection{\LaTeX{} Styling}

\begin{itemize}
	\item Try to avoid fancy \LaTeX formatting, as PlasTeX (used for generating HTML) is temperamental.  Especially macros don't always work, and sometimes the result is just ugly.
	\item We have the option to use different files for PDF and HTML generation, but we would generally prefer not to do this.  Look at \texttt{bsymb.sty} and \texttt{plastex-bsymb.sty} as an example. \textbf{NOTE:} We don't do this any more for any style files.
	\item Every section should have a label, reflecting the section name, all lowercase, spaces replaced with underscores (\_).
	\item Don't create subdirectories in the \texttt{latex} folder, as the scripts cannot always deal with them.
	\item Put images in the \texttt{img} folder.  Feel free to create additional directory structures underneath.
	\item Files other than images (e.g. Event-B projects) TODO - we have to figure out whether to keep them in svn (then they won't be accessible from PDFs) or on the Wiki (then they won't be accessible offline).
	\item Don't use hyperlinks for cross-references, but linked section numbers (generated with \texttt{\\ref{}}.  This is necessary for the print documentation to be useful.
	\item When including images in Latex, do not provide a width!  Instead, try to embed the print size in the image itself.  For instance, PNGs allow you to set the print size (in mm).  This way we can be sure that the images are rendered as HTML without distortion.
\end{itemize}



\chapter{Tutorial}
\label{tutorial}

This tutorial should provide the user with a tour through the most important functionalities of RODIN, so that he gets a understanding of how the program works.

The tutorial doesn't contain all the knowledge that you require.  Instead, it touches upon every concept - from installation to set theory to modeling and refinement - and helps you to find gaps in your knowledge.

Before we build a first model, we will cover some basic math.

\section{Tutorial Proposal (WP1-4)}

These are the chapters of the tutorial.  Average time available to write each chapter: 5 hours.  This is not much time.  Therefore, make sure that the sceleton exists before filling in details.  This is particularly true for screenshots: By all means indicate where screenshots should be, but don't waste time on them until the end.

\begin{description}
	\item[Background before getting started] -- we give a brief description of what Event-B is and what it is being used for; what kind of background knowledge we expect.
	\item[Installation] -- Guiding the user through downloading, installing and starting Rodin; Point out platform differences.  Install the provers.  Name the visible views and describe what they are doing.
	\item[Hello, world] -- a first machine, e.g. a traffic light with booleans for signals.  We introduce guards, resulting in the proof obligations to be discharged automatically. We explain how proof lables are read, without changing to the proof perspective.
	\item[Mathematical notation] -- At this point we quickly go through the most important aspects of predicate calculus and provide pointers to the reference and to external literature
	\item[Introducing Contexts] -- a little context-only example, but not too mathematical.  E.g. a data structure with axioms to show that it is a tree. Point out partitions as a typical pitfall (also add to FAQ)
	\item[Combining contexts and machines] -- introducing the ``sees'' relationship.
	\item[Refinement] -- introducing a refinement to add a new feature - the call button, for instance.
	\item[Theory behind refinement] -- describe what refinement means mathematically.
	\item[Data refinement] -- introduce data refinement (boolean to traffic light colors
	\item[Witnesses] -- What they are and how to work with them.  We need an example
	\item[Proving] -- hopefully, so far all proof obligations were discharged automatically.  Now we switch for the first time to the prooving perspective and explore it.  We change the auto prover configuration, invalidate proofs and show, that with the new configuration they don't discharge any more.  We prove a simple proof by hand.  We describe the provers available.
	\item[Tricky Proving] -- we start with a new example that contains a difficult proof.  We walk the user through discharging the proof with plenty of screen shots.
	\item[Complete Abrial Example] -- we pick an interesting example from the Abrial book, if we get permission.
	\item[Outlook] -- This concludes the tutorial, but we will provide many pointers to the user.  In particular, we will point to the literature from the deploy project, the Wiki and to plugins that solve specific problems.
\end{description}

\section{The First Machine: A Traffic Light Controller}
\label{tutorial_03}

% a first machine, e.g. a traffic light with booleans for signals.  We introduce guards, resulting in the proof obligations to be discharged automatically. We explain how proof lables are read, without changing to the proof perspective.

\tick{\textbf{Goals:} The objective of this section is to get acquainted with the modeling environment. We will create a very simple model consisting of just one file to develop a feeling for Rodin and Event-B.}

In this tutorial, we will create a model of a traffic light controller.  We will use this example repeatedly in subsequent sections.  Figure \ref{fig_tut_03_traffic_light} depicts what we are trying to achieve.

\begin{figure}[!h]
\begin{center}
	\includegraphics[]{img/tutorial/tut_03_trafficlight.png}
	\caption{The traffic light controller}
	\label{fig_tut_03_traffic_light}
\end{center}
\end{figure}

In this section, we will implement a simplified controller with the following characteristics:
\begin{itemize}
	\item We model the signals with boolean values to indicated ``stop'' (false) and ``go'' (true).  We do not model colors (yet).
	\item We do not model the push button yet.
\end{itemize}

\subsection{Project Setup}

Models typically consist of multiple files that are managed in a project (\ref{project}).  Create a new Event-B Project \textsf{File $\rangle$ New $\rangle$ Event-B Project}.  Give the project the name \texttt{tutorial-03} as shown in figure \ref{fig_tut_03_new_project_wizard}.

\begin{figure}[!h]
\begin{center}
	\includegraphics[]{img/tutorial/tut_03_tutorial-3.png}
	\caption{New Event-B Project Wizard}
	\label{fig_tut_03_new_project_wizard}
\end{center}
\end{figure}

\warning{Eclipse supports different types of projects.  The project must have the Rodin Nature (\ref{rodin_nature}) to work.  A project can have more than one nature.}

Next, create a new Event-B Component (\ref{eventb_component}).  Either use \textsf{File $\rangle$ New $\rangle$ Event-B Component} or right-click on the newly created project and select \textsf{New $\rangle$ Event-B Component}.  Use \texttt{mac} as the component name and click \textsf{Finish} as shown in figure \ref{fig_tut_03_new_component_wizard}. This will create a Machine (\ref{machine}) file.

\begin{figure}[!h]
\begin{center}
	\includegraphics[]{img/tutorial/tut_03_mac.png}
	\caption{New Event-B Component Wizard}
	\label{fig_tut_03_new_component_wizard}
\end{center}
\end{figure}

The newly created Component will open in the structural editor (\ref{structural_editor}).  The editor has four tabs at the bottom.  The \textsf{Pretty Print} shows the model as a whole with color highlighting, but it cannot be edited here.  This is useful to inspect the model.  The \textsf{Edit} allows editing of the model.  It shows the six main sections of a machine (REFINES, SEES, etc.) in a collapsed state.  You can click on the little triangle to the left of a section to expand it.

%\begin{figure}[!h]
%\begin{center}
%	\includegraphics[]{img/tutorial/tut_03_interface.png}
%\end{center}
%\end{figure}

The editor is \textit{form-based}.  This means that in well-defined places an appropriate control (text field, dropdown, etc.) allows modifications.

\info{Alternative editors are available as plug-ins.  The form editor has the advantage of guiding the user through the model, but it takes up a lot of space and can be slow for big models.  The text-based Camille Editor (\ref{camille}) is very popular.  Please visit the Rodin Wiki (\ref{rodin_wiki}) for the latest information.}


\subsection{Building the Model}

Back to the problem: Our objective is to build a simplified traffic light controller as described in \ref{tutorial_03}.  We start with the model state.  Two traffic lights will be modelled and we will therefore create two variables called  \texttt{cars\_go} and \texttt{peds\_go}.  Go to the \textsf{Edit} tab in the editor and expand the \textsf{VARIABLES} section.  Click on the green plus-sign to create a new variable.

\subsubsection{Creating Variables}

You will see two fields. The left one is filled with the word \texttt{var1}.  Change this to \texttt{cars\_go}.  The second field (after the double-slash ``//'') is a comment field in which you can write any necessary notes or explanations.

\info{\textbf{Comments:} The comment field supports line breaks.  Note that it is not possible to ``comment out'' parts of the model, as you would expect in any programming language.  You can use the comment field to ``park'' predicates and other strings temporarily.}

Create the second variable (\texttt{peds\_go}) in the same way.

%\begin{figure}[!h]
%\begin{center}
%	\includegraphics[]{img/tutorial/tut_03_new-variable.png}
%\end{center}
%\end{figure}

Upon saving, the variables will be highlighted in red, indicating an error as shown in figure \ref{fig_tut_03_error}.  The \textsf{Rodin Problems} view (\ref{rodin_problems_view}) shows corresponding error messages. In this case, the error message is ``Variable cars\_go does not have a type''.

\begin{figure}[!h]
\begin{center}
	\includegraphics[]{img/tutorial/tut_03_error.png}
	\caption{Red highlighted elements indicate errors}
	\label{fig_tut_03_error}
\end{center}
\end{figure}

Types are provided by invariants. Expand the \textsf{INVARIANTS} section and add two elements by following the same steps as above.  Invariants have labels (\ref{label}).  Default labels are generated (\texttt{inv1} and \texttt{inv2}).  The actual invariant is prepopulated with $\btrue$, which represents the logical value ``true''.  Change the first invariant to $cars\_go \in  BOOL$ and the second invariant to $peds\_go \in  BOOL$.  Event-B provides the build-in datatype \texttt{BOOL} amongst others (\ref{datatypes}).

%\begin{figure}[!h]
%\begin{center}
%	\includegraphics[width=0.7\textwidth]{img/tutorial/tut_03_invariants.png}
%\end{center}
%\end{figure}

\info{\textbf{Mathematical Symbols:} Every mathematical symbol has an ASCII-representation and the substitution occurs automatically.  To generate ``element of'' ($\in$), simply type a colon (``:'').  The editor will perform the substitution after a short delay. The \textsf{Symbols} view shows all supported mathematical symbols. The ASCII representation of a symbol can be found by hovering over the symbol in question.}

After saving, you should see that the \textsf{EVENTS} section is highlighted in yellow as demonstrated in Figure \ref{fig_tut_03_warning}.  Again, the \textsf{Rodin Problems} view gives us the reason: ``Variable cars\_go is not initialized''. Every variable must be initialized in a way that is consistent with the model.

\begin{figure}[!h]
\begin{center}
	\includegraphics[]{img/tutorial/tut_03_yellow.png}
	\caption{Yellow highlighted elements indicate warnings}
	\label{fig_tut_03_warning}
\end{center}
\end{figure}

To fix this problem, expand \textsf{EVENTS} and in turn the INITIALIZATION event (\ref{initialization}).  Add two elements in the \textsf{THEN} block.  These are actions that also have labels.  In the action fields, provide $cars\_go :=  FALSE$ and $peds\_go :=  FALSE$.

%\begin{figure}[!h]
%\begin{center}
%	\includegraphics[]{img/tutorial/tut_03_events.png}
%\end{center}
%\end{figure}

\subsubsection{State Transitions with Events}

Our traffic light controller cannot yet change its state.  To make this possible, we provide events (\ref{event}). We start with the traffic light for the pedestrians, and we will provide two events, one to set it to ``go'' and one to set it to ``stop''.

\warning{From now on, we won't describe the individual steps in the editor any more.  Instead, we will simply show the resulting model.} 

The two events will look as follows:

\pencil{
\begin{description}
	\EVT {set\_peds\_go}
		\begin{description}
		\BeginAct
			\begin{description}
			\nItemX{ act1 }{ peds\_go :=  TRUE }
			\end{description}
		\EndAct
		\end{description}
	\EVT {set\_peds\_stop}
		\begin{description}
		\BeginAct
			\begin{description}
			\nItemX{ act1 }{ peds\_go :=  FALSE }
			\end{description}
		\EndAct
		\end{description}
\end{description}
}

\subsubsection{Event parameters}

For the traffic light for the cars, we will present a different approach and use only one event with a parameter.  The event will use the new traffic light state as the argument.  The parameter is declared in the \textsf{any} section and typed in the \textsf{where} section:

\pencil{
\begin{description}
	\EVT {set\_cars}
		\begin{description}
		\AnyPrm
			\begin{description}
			\ItemX{ new\_value }
			\end{description}
		\WhereGrd
			\begin{description}
			\nItemX{ grd1 }{ new\_value \in  BOOL }
			\end{description}
		\ThenAct
			\begin{description}
			\nItemX{ act1 }{ cars\_go :=  new\_value }
			\end{description}
		\EndAct
		\end{description}
\end{description}
}

Note how the parameter is used in the action block.

\subsubsection{Invariants}
\label{tutorial:invariants}

If this model would control a traffic light, we would have a problem, as nothing is preventing the model from setting both traffic lights to \texttt{TRUE}.  The reason is that so far we only modeled the domain (the traffic lights and their states) and not the requirements.  We have the following safety requirement:

\begin{center}REQ-1: Both traffic lights must not be \texttt{TRUE} at the same time.\end{center}

We can model this requirement with the following invariant (please add this invariant to the model):
\[
\lnot  (cars\_go = TRUE \land  peds\_go = TRUE)
\]

Obviously, this invariant can be violated, and Rodin can tell us that.  The Explorer (\ref{eventb_explorer}) provides this information in various ways.  Go to the explorer and expand the project (\texttt{tutorial-03}), the machine (\texttt{mac}) and the entry ``Proof Obligations''.  You should see four proof obligations, two of which are not discharged.

\info{Make sure that you understand the proof obligation labels (\ref{po_labels}).  Also, the proof obligations can also be found via other entries.  Elements that have non-discharged proof obligations as children are marked with a small question mark.  For instance, \texttt{inv3} has all proof obligations as children, while the event \texttt{set\_cars} has one.}

To prevent the invariant from being violated (and therefore to allow all proof obligations to be discharged), we need to strengthen the guards (\ref{guard}) of the events.

\warning{Before looking at the solution, try to fix the model yourself.}

This concludes the tutorial.

\subsubsection{Finding Invariant Violations with ProB}
\begin{rodin-plugin}{img/prob.png}{ProB}
A useful tool for understanding and debugging a model is a model checker like ProB.  You can install ProB from the ProB Update Site, directly from ProB.

TODO: How to use.

\end{rodin-plugin}



\subsection{The Final Traffic Light Model}

\pencil{
\begin{description}
\MACHINE{mac}
\VARIABLES
	\begin{description}
		\Item{ cars\_go }
		\Item{ peds\_go }
	\end{description}
\INVARIANTS
	\begin{description}
		\nItemX{ inv1 }{ cars\_go \in  BOOL }
		\nItemX{ inv2 }{ peds\_go \in  BOOL }
		\nItemX{ inv3 }{ \lnot  (cars\_go = TRUE \land  peds\_go = TRUE) }
	\end{description}
\EVENTS
	\INITIALISATION
		\begin{description}
		\BeginAct
			\begin{description}
			\nItemX{ act1 }{ cars\_go :=  FALSE }
			\nItemX{ act2 }{ peds\_go :=  FALSE }
			\end{description}
		\EndAct
		\end{description}
	\EVT {set\_peds\_go}
		\begin{description}
		\WhenGrd
			\begin{description}
			\nItemX{ grd1 }{ cars\_go = FALSE }
			\end{description}
		\ThenAct
			\begin{description}
			\nItemX{ act1 }{ peds\_go :=  TRUE }
			\end{description}
		\EndAct
		\end{description}
	\EVT {set\_peds\_stop}
		\begin{description}
		\BeginAct
			\begin{description}
			\nItemX{ act1 }{ peds\_go :=  FALSE }
			\end{description}
		\EndAct
		\end{description}
	\EVT {set\_cars}
		\begin{description}
		\AnyPrm
			\begin{description}
			\ItemX{ new\_value }
			\end{description}
		\WhereGrd
			\begin{description}
			\nItemX{ grd1 }{ new\_value \in  BOOL }
			\nItemX{ grd2 }{ new\_value = TRUE \limp  peds\_go = FALSE }
			\end{description}
		\ThenAct
			\begin{description}
			\nItemX{ act1 }{ cars\_go :=  new\_value }
			\end{description}
		\EndAct
		\end{description}
\END
\end{description}
}






\chapter{Reference}

\section{Arithmetic}
\label{arithmetic}

\section{Eclipse}
\label{eclipse}

... Eclipse Definition ...

\section{Event-B}
\label{event-b}

Event-B is a formal method (\ref{formal_method}) for system-level modelling and analysis. Key features of Event-B are the use of set theory (\ref{set_theory}) as a modelling notation, the use of refinement (\ref{refinement}) to represent systems at different abstraction levels and the use of mathematical proof to verify consistency between refinement levels.

\paragraph{See Also:}
\begin{itemize}
\item \url{http://www.event-b.org}
\end{itemize}

\section{First Order Predicate Calculus}
\label{first_order_predicate_calculus}


\section{Formal Method}
\label{formal_method}

\section{IDE}
\label{ide}

... IDE Definition ...

\section{Naming Convention}
\label{naming_convention}

In this section we describe a recommended naming convention.  Good naming conventions save time -- and nerves.

\section{Refinement}
\label{refinement}

... Refinement Definition ...

\subsection{Horizontal Refinement}
\label{horizontal_refinement}

\subsection{Vertical Refinement}
\label{vertical_refinement}

\subsection{Data Refinement}
\label{data_refinement}

\paragraph*{See also:}
\begin{itemize}
\item Data refinement in the trafficlight tutorial (\ref{tutorial_tl_data_refinement})
\end{itemize}

\section{Propositional Calculus}
\label{propositional_calculus}

\section{Rodin}
\label{rodin}

... Rodin Definition ...

\section{Rodin Platform}
\label{rodin_platform}

... Rodin Platform Definition ...

\section{Set Theory}
\label{set_theory}

... Set Theory Definition ...





\chapter{Frequently Asked Questions}
\label{faq}

\section{General Questions}

\subsection{Where can I get help?}
\label{faq_getting_help}
\index{mailing list}

In addition to this handbook, consider looking in the Rodin Wiki (\ref{rodin_wiki}) for an answer.

There is also a vibrant community that is helpful and responsive.  You can contact it via the Rodin user mailing list at \href{mailto:rodin-b-sharp-user@lists.sourceforge.net}{rodin-b-sharp-user@lists.source\-forge\-.net}.

\subsection{What is Event-B?}
\index{Event-B}

Event-B is a formal method for system-level modelling and analysis. Key features of event-B are the use of set theory as a modelling notation, the use of refinement to represent systems at different abstraction levels and the use of mathematical proofs to verify consistency between refinement levels.
More details are available at \url{http://www.event-b.org}.

\subsection{What is the difference between Event-B and the B method?}

Event-B (\ref{tut_eventb}) is derived from the \href{http://en.wikipedia.org/wiki/B-Method}{B method}. Both notations have the same \href{http://en.wikipedia.org/wiki/Jean-Raymond_Abrial}{inventor}, and share many common concepts (set-theory, refinement, proof obligations, etc.). However, they are used for quite different purposes. The B method is devoted to the development of \textit{correct by construction} software, while the purpose of Event-B is used to model full systems (including hardware, software and environment of operation).

Event-B and the B method use mathematical languages which are similar but do not match exactly (in particular, operator precedences are different).

\subsection{What is Rodin?}
\index{Rodin}

The \textbf{Rodin Platform} is an Eclipse-based IDE for Event-B that provides support for refinement and mathematical proofs. The platform is open source, contributes to the Eclipse framework and can be extended with plugins.

\subsection{Where does the Rodin name come from?}

The Rodin Platform (\ref{rodin_platform}) was initially developed within the European Commission funded Rodin project (IST-511599), where Rodin is an acronym for ``Rigorous Open Development Environment for Complex Systems''. Rodin is also the name of a famous French sculptor. One of his most famous works is the \href{http://en.wikipedia.org/wiki/The_Thinker}{Thinker}. 

\subsection{Where I can download Rodin?}
\label{faq_where_download_rodin}

Rodin is available for download at the Rodin Download page: \url{http://wiki.event-b.org/index.php/Rodin_Platform_Releases}

\subsection{How to contribute and develop?}

Glad to hear that you want to help!  Please see the \url{http://wiki.event-b.org/index.php/Developer_FAQ} page.

\subsection{My operating system is not supported!  How can I install Rodin on my platform?}
\label{faq_os_not_supported}
At the time of writing this document, prebuild versions exist for only a small number of operating systems.  There are two recommended approaches for running Rodin in these situations:
\begin{description}
	\item[Build Rodin from the sources] Users who have some experience in building Java software can simply build Rodin from source.  For more information, please consult the Developer Documentation in the Rodin Wiki: \url{http://wiki.event-b.org/index.php/Rodin_Developer_Support}
	\item[Run Rodin in a virtual environment] With a fast computer, you can also use a virtual environment (e.g. VirtualBox) and install an operating system into that environment that supports Rodin (e.g. a 32bit version of Linux).
\end{description}

There are other options available for more specialized scenarios (e.g. running 32bit Rodin on a 64bit Linux system).  However, the two approaches described above are the most simple.

\section{General Tool Usage}

\subsection{Do I lose my proofs when I clean a project?}
\index{project!clean}
No! This is a common misunderstanding of what a project clean does. A project contains two kinds of files: 

\begin{itemize}
	\item those you can edit: contexts, machines, proofs 
	\item those generated by a project build: proof obligations, proof statuses (each proof obligation is either discharged or not discharged) 
\end{itemize}

The cleaner just undoes what the builder does, i.e. it removes proof obligations and statuses, but it never modifies any proof.

A status may change from \emph{discharged} to \emph{not discharged} when the proof is no longer compatible with the corresponding proof obligation (e.g. when a hypothesis is changed), but \textbf{the proof itself is still there!}
You can try to \href{http://wiki.event-b.org/index.php/Proof_Obligation_Commands#Proof_Replay_on_Undischarged_POs}{replay} it.

Confusion may arise when automatic provers have been launched. The cleaner does not undo these automatic proofs (why would it ?!!). Once a proof has been made, the platform does not modify or delete it by itself. Even obsolete proofs are preserved (\ref{obsolete_proof})!

\subsection{How do I install external plugins without using the Eclipse Update Manager?}

Although it is recommended that you install additional plugins into the Rodin platform using the Eclipse Update Manager, this might not always be practical. In this case, you can install these plugins by emulating the operations normally performed by the Update Manager either manually or by using ad-hoc scripts. 

The manual installation of plugins is described in \href{http://wiki.event-b.org/index.php/Installing_external_plugins_manually}{\emph{Installing external plugins manually}}. 

\subsection{The builder takes too long}

Generally, the builder spends most of its time attempting to prove POs. There are basically two ways to shorten this process: 

\begin{itemize}
	\item Disable the automated prover in the \textsf{Preferences} panel. 
	\item Mark a PO as reviewed if you do not want the auto-prover to attempt it anymore. 
\end{itemize}

Note that if you disable the automated prover, you always can run it later on some files by using the contextual menu in the Event-B Explorer. 

To disable the automated prover, open \textsf{Rodin Preferences} 
(menu \textsf{Window $\rangle$ Preferences...}). In the tree on the left-hand panel, select \textsf{Event-B $\rangle$ Sequent Prover $\rangle$ Auto/Post-tactic}. Then, in the main panel ensure that the checkbox labelled \textsf{Enable auto-tactic for proving} for proving is not selected. 

To review a proof obligation, just open it in the interactive prover and then click on the \emph{review} button (this is a round blue button with a \emph{R} in the \textsf{Proof Control} toolbar). The proof obligation should now labelled with the same icon in the Event-B explorer. 

\subsection{What are the ASCII shortcuts for mathematical operators?}
\index{symbols}

A page describing the ASCII shortcuts that can be used for entering mathematical operators can be found in the \textsf{Help} menu. To view this page, select \textsf{Help $\rangle$ Help Contents} and then select \textsf{Rodin Keyboard User Guide $\rangle$ Getting Started $\rangle$ Special Combos} in the window that pops up. 

This page is also available in the dynamic help system. The advantage of using dynamic help is that it is able to display the help page side-by-side with the other views and editors. To start the dynamic help, click \textsf{Help $\rangle$ Dynamic Help}, then select \textsf{Contents} and select the page in the tree. 

\subsection{Pretty Print does not work on Linux}
\index{xulrunner}

Configuring Rodin on Linux can be tricky.  In particular, the pretty print view of the original editor requires an HTML control to render.  It it does not work after installing Rodin, you may have to configure xulrunner as follows:

Add a property by appending the following code to your \textsf{eclipse/eclipse.ini} or \textsf{rodin/rodin.ini} file: 

\begin{verbatim} 
	-Dorg.eclipse.swt.browser.XULRunnerPath=
	  /usr/lib/xulrunner/xulrunner-xxx 
\end{verbatim} 

\subsection{Some mathematical characters are wrong}
\index{Font}

The Rodin editor must use the correct font to work properly, which is \textsf{Brave Sans Mono}. Depending on the editor, the font has to be configured via Window $\rangle$ Preferences $\rangle$ Colors and Fonts $\rangle$ Basic Text $\rangle$ Font.

\subsection{No More Handles}

On Windows platforms, Rodin may crash and generate the error message  ``no more handles''. An OS specific limitation is described \href{http://journals.jevon.org/users/jevon-phd/entry/19833}{here} and \href{https://bugs.eclipse.org/bugs/show_bug.cgi?id=211124}{here}. A workaround is provided at \href{http://blogs.msdn.com/b/ntdebugging/archive/2007/01/04/desktop-heap-overview.aspx}{this site}. 

\subsection{Software installation fails}

The installation of software from update sites (\textsf{Help $\rangle$ Install New Software...}) sometimes fails with an error saying something like: 

\begin{verbatim}
No repository found containing: 
        osgi.bundle,org.eclipse.emf.compare,1.0.1.v200909161031
No repository found containing: 
        osgi.bundle,org.eclipse.emf.compare.diff,1.0.1.v200909161031
...
\end{verbatim}


This is an eclipse/p2 bug that is referenced \href{http://stackoverflow.com/questions/511367/error-when-updating-eclipse}{here}. 

To fix this problem: 

\begin{itemize}
	\item Go to \textsf{Window $\rangle$ Preferences $\rangle$ Install/Update $\rangle$ Available Software Sites} 
	\item Remove all of the sites and then add them back again. This can be achieved in the \textsf{Available Software Sites} preference page by: 
	\begin{itemize}
		\item Selecting all  of the update sites (highlighting all those that are checked) 
		\item Exporting them 
		\item Removing them
		\item Restarting Rodin
		\item Going back to the preference page and importing the update sites back (from the previously exported file) 
	\end{itemize}
\end{itemize}

\subsection{How do I send a bug report?}
\label{bug_report}

This depends on the nature of the bug:
\begin{itemize}
	\item Problems with the core Rodin platform, as well as feature requests, should be filed via the SourceForge bug tracker: \url{http://wiki.event-b.org/index.php/Bugs_and_Feature_Requests}
	\item To file problems with individual plugins, check the plugin's documentation in the wiki (\ref{rodin_wiki}).
	\item If you are unsure whether to file a bug or not, consider asking a question on the Rodin user list at \href{mailto:rodin-b-sharp-user@lists.sourceforge.net}{rodin-b-sharp-user@lists.sourceforge.net}.
	\item To report a problem with the handbook, use the feedback button that is present in the HTML and Eclipse Help version of the handbook.
\end{itemize}

\subsection{Where did the GUI window go?}

When you are looking for a particular view, and the view does not appear or if it appears in a different place than is usual, try clicking on \textsf{Window $\rangle $ Reset Perspective...}. This will reset the different views back to their default positions. If you can't find menu buttons from one of the views, try resizing the view in question to see if part of the menu has been hidden.

\subsection{Where vs. When: What's going on?}
\index{when}
\index{where}

You may have noticed that both in this tutorial, as well as in the tool, events sometimes use the keyword ``when'' and sometimes ``where''.  The idea of this was to make the formal statements more intuitive.  Unfortunately, this created more confusion than anything else.

The short answer is: ``when'' and ``where'' in events have exactly the same meaning, for all practical purposes.

The long answer is: In some contexts (but not all), the tool changes the keywords to make the meaning of the event more apparent.  The distinguishing factor is the parameter: an event without a parameter uses the keyword ``when'', and an event with a parameter uses the keyword ``where''.

To make things even more confusing, this doesn't apply everywhere: The Event-B structural editor always uses the keyword ``where'', but the pretty print for the Event-B structural editor switches between the two.  The default Rodin editor always uses the keyword "where". The Event-B syntax in this handbook has been generated with the \LaTeX plugin, which also switches between the two keywords.

\section{Modelling}
\index{modelling}

\subsection{Witness for \textsf{Xyz} missing. Default witness generated}

A parameter has disappeared during a refinement. If this is intentional, add a witness (\ref{witness}) to tell the machine how the abstract parameter should refined. 

\subsection{Identifier \textsf{Xyz} should not occur free in a witness}

This means that the \textsf{Xyz} identifier appears in a witness predicate, but \textsf{Xyz} is a disappearing abstract variable or parameter and is not set as the witness label. To resolve this error, set change the witness label to the identifier \textsf{Xyz}.

\subsection{Witness \textsf{Xyz} must be a disappearing abstract variable or parameter in the \textsf{INITIALISATION} event}
\index{witness}

The witness is for the after value of the abstract variable, hence you should use the primed variable. The witness label should be \textsf{Xyz'}, and the predicate should refer to \textsf{Xyz'} too. 

\subsection{I've added a witness for \textsf{Xyz} but it keeps saying ``Identifier \textsf{Xyz} has not been defined''}

As specified in the Section \ref{witness}, the witness must be labelled with the name \textsf{Xyz} of the abstract parameter of the event that is being refined. A concrete example can be found in Section \ref{tut_extend_traffic_witnesses}.

\subsection{How can I create a new Event-B Project?}
\index{project}

Please read Section \ref{tut_project_setup} to learn how to create a new Event-B project.

\subsection{How can I remove a Event-B Project?}

In order to remove a project, first select it on the \textsf{Project Explorer} and then right click with the mouse. The contextual menu will appear on the screen as indicated in Figure \ref{fig_faq_removeproject}.

\begin{figure}[!ht]
\begin{center}
	\includerodinimg{faq/faq_removeproject.png}
	\caption{Removing a Event-B Project}
	\label{fig_faq_removeproject}
\end{center}
\end{figure}

Simply click on \textsf{Delete}, and your project will be deleted (after you confirm that you want to delete it in the window that pops up). It is then removed from the \textsf{Project Explorer}.

\subsection{How can I export an Event-B Project?}

Exporting a project is the operation by which you can construct automatically a ``.zip" file containing the entire project. Such a file can be sent by email. Once received, an exported project can be imported (next section). It then becomes a project like the other ones which were created locally. In order to export a project, select it and then select on \textsf{File $\rangle$ Export...} from the menubar as indicated in Figure \ref{fig_faq_exportproject}. 

\begin{figure}[!ht]
\begin{center}
	\includerodinimg{faq/faq_exportproject.png}
	\caption{Export a Event-B Project}
	\label{fig_faq_exportproject}
\end{center}
\end{figure}

The \textsf{Export} wizard will pop up. In this window, select \textsf{General $\rangle$ Archive File} and click the \textsf{Next $>$} button. Specify the path and name of the archive file into which you want to export your project and finally select \textsf{Finish}. This menu sequence (and the various options) is a part of the Eclipse platform. For more information, refer to the Eclipse documentation. 

\subsection{How can I import a Event-B Project?}

A ``.zip" file corresponding to a project which has been exported elsewhere can be imported locally. In order to do this, click on \textsf{File $\rangle$ Import} from the menubar. In the import wizard, select \textsf{General $\rangle$ Existing Projects into Workspace} and click \textsf{Next $>$}. Then choose the \textsf{Select archive file} option and hit the \textsf{Browse...} button to find the zip file that you want to import. Now click \textsf{Finish}. As with exporting, this menu sequence and layout are part of Eclipse.

The importation will fail if the name of the imported project (not the name of the file) is the same as the name of an existing local project. This means that when you are exporting a project, it is a good idea to modify its name in case the person who imports the project already has a project with that same name (which could be a previous version of the exported project). Changing the name of a project is explained in the next section. 

\subsection{How can I change the name of a Event-B Project?}

Select the project whose name you want to modify, and then click on \textsf{File $\rangle$ Rename...}. Modify the name and click on \textsf{OK}. The name of your project will then have been modified accordingly. 

\subsection{How can I create a Event-B Component?}

Please check Section \ref{tut_project_setup} to learn how to create a new Event-B component.

\subsection{How can I remove a Event-B Component?}

In order to remove a component, press the right-click on the component. In the context menu, select \textsf{Delete}. This component is removed from the \textsf{Project Explorer}. 

\subsection{In the new Rodin Editor, how can I add an element to machine?}
\label{faq_new_editor_new_element}

\info{Please also consult Section~\ref{new_eventb_editor}. The editor is described in more detail there.} 


Whenever you pull up a context menu in the new editor, please pay attention to the following two issues:
\begin{itemize}
	\item Make sure that the cursor already is on the correct line.  If you right-click and the cursor is on the wrong line or in the wrong position within the line, you will get an incorrect context menu.
	\item Make sure the cursor is not in ``edit'' mode. This is the case when you are able to edit a textual element.  If this is the case, you will also get an incorrect context menu.
\end{itemize}

The different elements of the machine, can of course, be added using the different wizards for element creation (New Variable Wizard \icon{rodin/newvar_edit.png}, New Variant Wizard \icon{rodin/newvariant_edit.png}, New Invariant Wizard \icon{rodin/newinv_edit.png}, and New Event Wizard \icon{rodin/newevt_edit.png}) which are described in more detail in Section \ref{eventb_wizards}. 

You can also add new elements by placing your cursor directly to the left of the small green arrow (\icon{rodin/structured_arrow.png}) that appears next to your machine name in \textsf{MACHINE} section. Now right click and select the component that you want to add from the \textsf{Add Child} menu. You can also add an element by right clicking on the heading of the section of the element you want to add (e.g. \textsf{VARIABLES}) and selecting \textsf{Add Child}, or by placing your cursor directly to the left of the small green arrow (\icon{rodin/structured_arrow.png}) next to the name of any of the components that already exist and selecting \textsf{Add Sibling}. Unfortunately, if your cursor is not directly next to the small green arrow (while the cursor is blinking, the left side of the arrow is actually touching the cursor), these methods do not actually work. 

\subsection{How can I use multiple lines for a comment, predicate or expression (using the new editor)?}

To insert a line break while editing any field, use Ctrl-Return.

\subsection{How can I save a Context or a Machine?}

Once a machine or context is (partially) edited, you can save it by using the save button as indicated in Figure \ref{fig_faq_saveaction}.

\begin{figure}[!ht]
\begin{center}
	\includerodinimg{faq/faq_saveaction.png}
	\caption{Save a context or a machine}
	\label{fig_faq_saveaction}
\end{center}
\end{figure}

Once a ``Save" has been completed, three tools are called automatically, these are:

\begin{itemize}
	\item the Static Checker
	\item the Proof Obligation generator (\ref{generated_proof_obligations})
	\item the Auto-Prover (\ref{auto_prover})
\end{itemize}

This can take some time. A ``Progress'' view can be opened at the bottom right of the screen to see which tools are working (most of the time, it will be the auto-prover).  This is done via Window $\rangle$ Show View $\rangle$ Progress.

\section{Proving}
\index{proving}

\subsection{Help!  Proving is difficult!}

Yes, it is.  Check out Section~\ref{use_provers_effectively} to begin using the provers.

\subsection{How can I do a Proof by Induction?}
\index{induction@proof by induction}
\href{http://wiki.event-b.org/index.php/Induction_proof}{This page about proof by induction} will give you some starting tips.

\subsection{What do the labels on the proof tree mean?}

\begin{itemize}
	\item \textsf{ah} means \textit{add hypothesis},
	\item \textsf{eh} means rewrite with \textit{equality from hypothesis} from left to right,
	\item \textsf{he} means rewrite with \textit{equality from hypothesis} from right to left,
	\item \textsf{rv} tells us that this goal has been manually reviewed (\ref{proof_control_view}),
	\item \textsf{sl/ds} means \textit{selection/deselection},
	\item \textsf{PP} means \textit{discharged by the predicate prover},
	\item \textsf{ML} means \textit{discharged by the mono lemma prover}
\end{itemize}

%%% Local Variables: 
%%% mode: latex
%%% TeX-master: "rodin-doc"
%%% End: 


\clearpage
\addcontentsline{toc}{chapter}{Index} 
\printindex

\end{document}

