\section{Outlook}
\label{tutorial_11}

Congratulations -- if you made it this far, you should have a good foundation for getting some real work done with Rodin.  In the following we'd like to provide you with a few pointers that will help you to make your work as efficient as possible.

\begin{description}
	\item[Use the Reference Section and FAQ] If you have a specific issue or if you quickly need to look something up, check the reference (\ref{reference}) and FAQ (\ref{faq}) of this handbook.
	\item[There is an Online, PDF and an Eclipse-Version of this Handbook] There are three versions of this handbook.  You can access it directly through Rodin by using the built-in help browser (\textsf{Help $\rangle$ Help Contents}).  In particular, the Eclipse-Version can be used offline.
	\item[Use the Rodin Wiki] The Rodin Wiki (\ref{rodin_wiki}) contains the latest news regarding Rodin and a wealth of information that is not in the scope of this handbook.  Be sure to check out what's there.
	\item[Find useful Plugins] There are many plugins available, be sure to check them out.  There is a good chance that they will make your life easier.
	\item[Subscribe to the mailinglists] The wiki documents the existing mailinglists, which include a list for users and evelopers.  We strongly recommend to subscribe to the announcement list.
	\item[Rodin in Industry] If you consider using Rodin in Industry, be sure to explore the deliverables from the Deploy (\ref{deploy}) project, where industrial partners describe their experiences with Rodin.
\end{description}

We wish you success in your modeling projects!

