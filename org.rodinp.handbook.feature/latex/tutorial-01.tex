\section{Before Getting Started}
\label{tutorial_1}

Before we get started with the actual tutorial, we are going to go over the required background to make sure that you have a rudimentary understanding of the necessary concepts.

\tick{\textbf{You can skip this section, if...}
\begin{itemize}
	\item ... you know what formal modeling means
	\item ... you know what predicate logic is
	\item ... you know what Event-B, Rodin refer to
	\item ... you know what Eclipse is
\end{itemize}
}

\subsection{Systems Development}

Ultimately, the purpose of the methods and tools introduced here is to improve systems development.  By this we mean the design and management of complex engineering projects over their life cycle.  Examples include cars, air traffic control systems, etc.

``Taking an interdisciplinary approach to engineering systems is inherently complex since the behaviour of and interaction among system components is not always immediately well defined or understood. Defining and characterizing such systems and subsystems and the interactions among them is one of the goals of systems engineering. In doing so, the gap that exists between informal requirements from users, operators, marketing organizations, and technical specifications is successfully bridged.''\footnote{http://en.wikipedia.org/wiki/Systems\_engineering\#Managing\_complexity}

We are concerned with \textit{formalizing specifications}.  This allows us a more rigorous analysis (thereby improving the quality), and allows the reuse of the specification to develop an implementation.  This comes at the cost of higher up-front investments.

\subsection{Formal Modelling}



\subsection{Predicate Logic}

\subsection{Event-B}

Event-B is a notation for formal modelling based around an abstract machine notation (\ref{abstract_machine_notation}).

Event-B is considered an evolution of B (also known as classical B). It is a simpler notation, which is easier to learn and use. It comes with tool support in the form of the Rodin Platform.

\subsection{Rodin}

\subsection{Eclipse}

Rodin is based on the Eclipse Platform (\ref{eclipse}), a Java-based platform for building software tools.  This matters for two reasons:
\begin{itemize}
	\item If you have already used Eclipse-based software, then you will feel immediately comfortable with the handling of the Rodin application.
	\item There are many extensions available to Eclipse-based software, called plug-ins (\ref{plugin}).  There are Rodin-specific plugins as well as Rodin-independent plugins that may be useful to you.  The Rodin Wiki (\ref{rodin_wiki}), contains a list of plugins is maintained.
\end{itemize}

The GUI of an Eclipse application consists of Views, Editors, Toolbars, Quickview, Perspectives and many more elements.  If these terms are unfamiliar to you, please consult Section~\ref{eclipse} which contains pointers to Eclipse tutorials.

In Section~\ref{tutorial_2}, we present the Rodin-specific GUI elements.
