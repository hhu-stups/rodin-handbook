\chapter{Frequently Asked Questions}
\label{faq}

\section{General Questions}

\subsection{What is Event-B?}

Event-B is a formal method for system-level modelling and analysis. Key features of event-B are the use of set theory as a modelling notation, the use of refinement to represent systems at different abstraction levels and the use of mathematical proof to verify consistency between refinement levels.
More details are available in \url{http://www.event-b.org}.

\subsection{What is the difference between Event-B and the B method?}

Event-B (\ref{eventb}) is derived from the B method. Both notations have the same inventor, and share many common concepts (set-theory, refinement, proof obligations, ...) However, they are used for quite different purpose. The B method is devoted to the development of correct by construction software, while the purpose of event-B is to model full systems (including hardware, software and environment of operation).

Both notations use a mathematical language which are quite close but do not match exactly (in particular, operator precedences are different).

\subsection{What is Rodin?}

The Rodin Platform is an Eclipse-based IDE for Event-B that provides effective support for refinement and mathematical proof. The platform is open source, contributes to the Eclipse framework and is further extendable with plugins.

\subsection{Where does the Rodin name come from?}

The Rodin Platform (\ref{rodin_platform}) was initially developed within the European Commission funded Rodin project (IST-511599 ), where Rodin is an acronym for "Rigorous Open Development Environment for Complex Systems” . Rodin is also the name of a famous French sculptor, one of his most famous work being the Thinker. 

\section{Installation Questions}

\section{Proofer Questions}

\section{Usage Questions}

\subsection{Where vs. When: What's going on?}

\section{Plug-In Questions}
