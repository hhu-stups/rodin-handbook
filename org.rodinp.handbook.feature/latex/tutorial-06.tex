\section{Event-B Concepts}
\label{tut_eventb_concepts}

\tick{\textbf{Goals:} In this section we give you an understanding over the fundamental concepts
  of Event-B.}

In Event-B we have two kind of components.
A \emph{context} describes the static elements of a model. A \emph{machine} describes the dynamic behavior of a model. We have already used a machine to model the traffic light problem in Section \ref{tut_first_machine}. In the last section (\ref{tut_contexts}), we used a context to model the Agatha problem.

\subsection{Contexts}
\label{tut_concepts_contexts}
\index{context}

A context has the following components:
\begin{description}
\item[Sets]
\index{set}
  User-defined types can be declared in the \textsl{SETS} section
  (See also \ref{sets}).
\item[Constants]
\index{constant}
  We can declare constants here. The type of each constant must be declared in the axiom section.
\item[Axioms] The axiom section contains a list of predicates (called axioms).
\index{axiom}
  These axioms define rules that will always be the case for given elements of the context. These rules can then be taken for granted when developing a model.
  The axioms can be used later in proofs that for components that
  use (``\emph{see}'') this context.
  Each axiom has a label attached to it.
\item[Theorems]
\index{theorem} 
  Axioms can be marked as \emph{theorems}. If this is the case, we are declaring that
  the predicate can be proved by using the axioms that have been written 
  before this theorem.
  Once they have been proven, theorems can be used later in proofs just like the other axioms.
\item[Extends]
\index{extends}
  A context may extend an arbitrary number of other contexts.
  When we extend another context $A$, we can then use all constants and axioms declared in $A$ and also add new constants
  and axioms.
\end{description}
\index{proving!proof obligation}
Rodin automatically generates \emph{proof obligations} (often abbreviated as PO) for properties that need to be proven. Each proof obligation has a name that identifies where the proof obligation was generated.
There are two kind of proof obligations generated in a context:
\begin{itemize}
  \item Each theorem must be proven. The proof obligation's name has the form 
    \eventbpo{label/THM}, where \eventbpo{label} is the theorem's label.
  \item Some expressions are not obviously \emph{well-defined}. For example, the axiom $x \div y > 2$ is only meaningful if $y$ is different from 0.
    Thus Rodin generates the proof obligation $y\neq 0$.
    A proof obligation for proving than an expression is well-defined has the name \eventbpo{label/WD}.
\end{itemize}
The order of the axioms and theorems matter because the proof of a theorem or the degree to which an expression is well-definded may depend on the axioms and theorems that have already been written. This is necessary to avoid circular reasoning.


\subsection{Machines}
\label{tut_machines}
\index{machine}

A machine describes the dynamic behavior of a model by means of
variables whose values are changed by events.
A central aspect of modeling a machine is to prove that the machine
never reaches an invalid state, i.e. the variables always have values
that satisfy the invariants.
Here is a brief summary of the part that a machine contains:

\begin{description}
\item[Refines] \index{refines} A machine has the option of refining another one.
  We will see in Section \ref{tut_refinement} what that means.
\item[Sees] \index{sees} We can use the context's sets, constants and axioms  in
  a machine by declaring it in the \textsl{Sees} section.
  The axioms can be used in every proof in the machine as hypotheses.
\item[Variables] \index{variable}
  The variables' values are determined by an initialization event and
  can be changed by events. This constitues the state of the machine. The type of each variable must be declared in the invariant section.
\item[Invariants] \index{invariant}
  These are predicates that should be true for every reachable state.
  Each invariant has a label.
\item[Events] \index{event}
  An event can assign new values to variables.
  The \emph{guards} of an event specify the conditions under which it can be executed.
  The initialization of the machine is a special case of an event.
\end{description}

\subsection{Events}
\label{tut_events}
\index{event}

We saw in Section \ref{tut_first_machine} what an event basically looks like by using the example of a traffic light:
\begin{description}
	\EVT {set\_cars}
		\begin{description}
		\AnyPrm
			\begin{description}
			\ItemX{ new\_value }
			\end{description}
		\WhereGrd
			\begin{description}
			\nItemX{ grd1 }{ new\_value \in  BOOL }
			\end{description}
		\ThenAct
			\begin{description}
			\nItemX{ act1 }{ cars\_go :=  new\_value }
			\end{description}
		\EndAct
		\end{description}
\end{description}
We have the event's name $set\_cars$, a \emph{parameter} with the name $new\_value$,
a \emph{guard} with label grd1 and an \emph{action} with label act1.
An event can have an arbitrary number of parameters, guards and events.

The guards specify \emph{when} an event is allowed to occur, i.e. the event can only be executed if the values of the machine's variables and parameters match the values listed in the guard. If this is the case, we say that the event is enabled.
The actions describe \emph{what} changes will then be applied to the variables.

Only the variables that are explicitly mentioned in the actions are affected.
All the other variables keep their old values. Beside the simple assignment ($\bcmeq$),
there are other forms of actions ($\bcmin$ or $\bcmsuch$) which are explained in
the reference section \ref{actions}.

\index{initialization}
The \emph{initialization} of the machine is a special form of event. It has neither parameters
nor guards.

Invariants must always be valid. To verify this, we must prove two things:
\begin{itemize}
  \item The initialization leads to a state where the invariant is valid.
  \item Assuming that the machine is in a state where the invariant is valid,
    every enabled event leads to a state where the invariant is valid.
\end{itemize}

Rodin generates proof obligations for every invariant that can be affected by an event, i.e. the invariant contains variables that can be changed by an event.
The name of the proof obligation is then \\ \eventbpo{event\_name/invariant\_label/INV}.
The goal of such a proof is to assert that when all affected variables are replaced by new values from the actions, the invariant still holds. The hypotheses for such a proof obligation consist of:

\begin{itemize}

\item All invariants, because we assume that all invariants hold before the event is triggered,
\item All guards, because events can only be triggered when the guards are valid.
\end{itemize}

In the special case of an initialization event, we cannot use the invariants because we do not make any assumptions about uninitialized machines.

\subsection{Refinement}
\label{tut_refinement}
\index{refinement}
Refinement is a central concept in Event-B. Refinements are used to gradually
introduce the details and complexity into a model.
If a machine \texttt{B} refines a machine \texttt{A}, \texttt{B} can only behave in a way that
corresponds to the behavior of \texttt{A}. We will now look into more detail of what ``corresponds''
here means.
In such a setting, we call \texttt{A} the abstract and \texttt{B} the concrete machine.

This is just overview over the concept of refinement. Later in Section \ref{tut_expanding_traffic_light_system}
we will use refinement in an example.

The concrete machine has its own set of variables. Its invariants can refer to
the variables of the concrete and the abstract machine. If a invariant refers to both,
we call it a ``gluing invariant''. The gluing invariants are
used to relate the states between the concrete and abstract machines.

An event of the abstract machine may be refined by one or several events of the concrete machine.
To ensure that the concrete machine does only what is allowed to do by the abstract one, we must prove two things:
\begin{itemize}
\item The concrete events can only occur when the abstract one occurs.
\item If a concrete event occurs, the abstract event can occur in such a way that the resulting
  states correspond again, i.e. the gluing invariant remains true.
\end{itemize}

The first condition is called ``guard strengthening''. The resulting proof obligation has the label
\eventbpo{concrete\_event/abstract\_guard/GRD}. We have to prove that under the assumption that the
concrete event is enabled (i.e. its guard are true) and the invariants (both the abstract and the concrete) 
hold, the abstract guards holds as well. Thus the goal is to prove that the abstract guard, the invariants and the
concrete guards can be used as hypotheses in the proof.

The second condition, that the gluing invariant remains true, is just a more general case of the proof obligation which ensures 
that an event does not violate the invariant. So the proof obligation's label is again 
\eventbpo{concrete\_event/concrete\_invariant/INV}. The goal is to prove that the invariant of the concrete machine is valid when each occurence of a modified variable is replaced by its new value.
The hypotheses we use are:
\begin{itemize}
\item We assume that the invariant of both the concrete and abstract machines were valid before the event occurred.
\item The abstract invariants where the modified variables are replaced by their new values are valid because we know that the abstract event does not violate any invariants.
\item The event occurs only when the guards of both the concrete and abstract machines are true.
\end{itemize}

These two conditions are the central issues that we need to deal with to prove the correctness of a refinement. We now just explain a few
common special cases.

\subsubsection{Variable re-use}
\label{tut_variable_reuse}
Most of the time, we do not want to replace all variables with new ones. It is sometimes useful to keep all of the variables.
We can do this just by repeating the names of the abstract variables in the variable section of the concrete
machine. In that case, we must prove for each concrete event that changes this variable that the corresponding 
abstract event updates the variable in the same way.
The proof obligation has the name \eventbpo{concrete\_event/abstract\_action/SIM}.

\subsubsection{Introducing new events}
\label{tut_skip}
An event in the concrete machine might not refine any event in the abstract machine. In that case it is assumed
to refine \emph{skip}, which is the event that does nothing and can occur any time. 
The guard strengthening is then true and doesn't need to be proven.
We still have to prove that the gluing invariant holds but this time under the assumption that the abstract machine's
variables have not changed. Therefore, the new state of our newly introduced event corresponds to the same state of
our abstract machine from before the event happened.

\subsubsection{Witnesses}
\label{tut_witnesses}
\index{witness}
Let's consider a situation where we have an abstract event with a parameter $p$ and we are dealing with a refined event that no longer needs that parameter.
We saw above that we have to prove that for each concrete event the abstract event may act accordingly.
With the parameter, however, we now have the situation in which we must prove the existence of a value for $p$ such
that an abstract event exists. Proofs with existential quantification are often hard to do, so Event-b 
uses the a \emph{witness} construct. A witness is just a predicate of the abstract parameter with the
name of the variable as label. Often a witness has just the simple form $p = \ldots$.


%%% Local Variables: 
%%% mode: latex
%%% TeX-master: "rodin-doc"
%%% End: 
