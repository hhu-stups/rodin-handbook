\section{Mathematical Notation}
\label{reference_03}

Here we cover the complete mathematical notation of Event-B. For each expression (like an operator) we describe its purpose, its type, the type of its arguments and well-definedness-conditions. We roughly separate the expressions into three groups: predicates, set-theoretical and arithmetic.

Some laws (like the commutative law of addition in arithmetic) would be nice but will not be part of the first iteration of the documentation. E.g. the Z reference manual does this.

References to related proof rules would be nice to have, too. But again, this is not part of the first iteration.

\subsection{Introduction}

What data types exist, what are well-definedness-conditions, how the description of the expressions is organized.

\subsection{Predicates}

All operators that work with predicates ($\land$, $\lor$, quantifier, \ldots). 

\subsection{Sets and relations}

\subsection{Arithmetic}
