\section{Event-B (Modeling notation)}
\label{reference_02}

In Event-B, we have two types of components: contexts and models.
Here we describe briefly the different elements of a context or model.

Predicates or expressions can be occur at various places in both components.
For each predicate or expression, we briefly describe the scope, i.e. which
identifier can be used.

We explain which proof obligations are generated. 
For each proof obligation, we list the goal and the hypothesis that can be used.

\subsection{Contexts}
\index{Context}
A context describes the static part of a model. It consists of
\begin{itemize}
\item Sets
\item Constants
\item Axioms
\item Extended contexts
\end{itemize}

\subsubsection{Sets}
\label{sets}
A new type can be declared by adding its name -- an identifier -- to the \eventbsection{Sets} section.
The identifier must be unique, i.e. it must not have been already declared as a constant or set in an extended context.
The identifier then denotes the set of all elements of the type. 

A common pattern for declaring enumerated sets (sets where all elements are explicitly given)
is to use the partition operator. If we want to specify a set $S$ with elements $e_1,\ldots,e_n$, then
we declare $S$ as a set, $e_1,\ldots,e_n$ as constants and add the axiom $partition(S,e_1,\ldots,e_n)$.

\subsubsection{Constants and axioms}
\label{constants_and_axioms}
Constants can be declared by adding their unique name (an identifier) to the \eventbsection{Constants} section.
The type of the constant must be inferable by the axioms.

An axiom describes a statement that can be assumed as true in the rest of the model.
Each axiom consists of a label and a predicate.
In an axiom, constants and sets of the context itself and of extended contextes can be referenced.

An axiom can be marked as a theorem. In that case a proof obligation with the label \textsf{axiomlabel/THM} is added.
The goal of the proof obligation is the theorem itself.
Axioms that are declared before the theorem and the axioms of extended contexts can be used as hypotheses.

If an axiom contains any condition that is well-defined, a proof obligation of the form \textsf{axiomlabel/WD} is generated.
Axioms declared beforehand and axioms of extended contextes can be used as hypotheses.
   
\subsubsection{Extending a context}
Contexts can be extended by adding their name to the \eventbsection{Extends} section.

The resulting context consists of all constants and axioms of all extended contexts and the extending context itself.
Thus for a context/machine that extends/sees the contexts, it makes no difference where a constant or axiom is declared.

\subsection{Machines}
A machine describes the dynamic behaviour of a model by means of variables whose values are changed by events.

\subsubsection{Variables and invariants (with theorems), generated POs}
\index{Variable}
Variables can be declared by adding their unique name (an identifier) to the \eventbsection{Variables}.
The type of the variable must be inferable by the invariants of the machine.

An invariant describes a statement that should be hold in each state of the machine.
Each invariant consists of a label and a predicate.
Identifier that can be referenced in an invariant:
\begin{itemize}
\item Variables of the machine itself and of a refined machine (transitively, i.e. if the refined machine refines another one, that variables can be also used, and so on)
\item Constants and sets of seen contexts
\end{itemize}

An invariant can be marked as a theorem. In that case a proof obligation with the label (TODO: THM/label or something like that) is added.
The goal of the proof obligation is the theorem itself.
Hypothesis that can be used in the proof.
\begin{itemize}
\item Axioms of seen contexts
\item Invariants that are declared above the theorem
\item Invariants of refined machines (TODO: Check if this is true)
\end{itemize}

\subsubsection{Seeing a context}
If the model sees a context, 

\subsubsection{Refining a model}

Refines clause, gluing invariant, general picture of refinement with pointers to the event below, re-use of variables.

\subsubsection{Events}

All aspects of events are covered: Parameters, guards and actions, generated POs, refinement (again with generated POs, re-use of parameters, witnesses), status (normal/convergent/anticipated), merging of events.

\subsubsection{Termination}

How to prove termination by the model's variant and the status of events.
  
\subsection{Generated proof obligations}

We give a brief overview about what POs are generated where. This should help the user to identify the reason of a PO when he just know its label.


%%% Local Variables: 
%%% mode: latex
%%% TeX-master: "rodin-doc"
%%% End: 
