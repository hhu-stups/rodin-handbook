\section{Mathematical Notation}
\label{reference_03}

Here we cover the complete mathematical notation of Event-B. For each expression (like an operator) we describe its purpose, its type, the type of its arguments and well-definedness-conditions. We roughly separate the expressions into three groups: predicates, set-theoretical and arithmetic.

Some laws (like the commutative law of addition in arithmetic) would be nice but will not be part of the first iteration of the documentation. E.g. the Z reference manual does this.

References to related proof rules would be nice to have, too. But again, this is not part of the first iteration.

\subsection{Introduction}

What data types exist, what are well-definedness-conditions, how the description of the expressions is organized.

\subsection{Predicates}

All operators that work with predicates ($\land$, $\lor$, quantifier, \ldots). 

\subsection{Sets and relations}

\subsection{Arithmetic}

\subsubsection{Sets of numbers}
\begin{rodinrefnames}
  $\intg$  & \texttt{INT}  & Integers \\
  $\nat$   & \texttt{NAT}  & Natural numbers, starting with 0 \\
  $\nat_1$ & \texttt{NAT1} & Natural numbers, starting with 1 \\
  $\upto$  & \texttt{..}   & Range of numbers
\end{rodinrefnames}
\rodinrefsec{Description}
The set of all integers is denoted by $\intg$. It contains all elements of the type.
The two subsets $\nat$ and $\nat_1$ contain all elements greater or equal to 0 resp. 1.
The range of numbers between $a$ and $b$ is denoted by $a \upto b$. If $a$ is greater
than $b$, $a \upto b$ is empty.
\rodinrefsec{Definition}
$\nat   = \{~ n\in\intg~|~n\geq 0~\}$\\
$\nat_1 = \{~ n\in\intg~|~n\geq 1~\}$\\
$a\upto b = \{~ n\in\intg~|~a\leq n \land n\leq b~\}$
%\begin{flalign*}
%  \nat   &= \{~ n\in\intg~|~n\geq 0~\} &\\
%  \nat_1 &= \{~ n\in\intg~|~n\geq 1~\} &
%\end{flalign*}
\rodinrefsec{Types}
\begin{tabular}{l@{\,}l@{\qquad}l@{ }l}
  $\intg$ & $\in\pow(\intg)$ \\
  $\nat$  & $\in \pow(\intg)$ \\
  $\nat_1$ & $\in\pow(\intg)$ \\
  $a \upto b$ & $\in\pow(\intg)$ & with & $a\in\intg$\\
              &                  &      & $b\in\intg$\\
\end{tabular}

\subsubsection{Arithmetic operations}
\begin{rodinrefnames}
  $+$      & \texttt{+}   & Addition \\
  $-$      & \texttt{-}   & Subtraction or unary minus \\
  $\cdot$ & \texttt{*}   & Multiplication \\
  $\div$   & \texttt{/}   & Integer division \\
  $\mod$   & \texttt{mod} & Modulo \\
\end{rodinrefnames}
\rodinrefsec{Description}
These are the usual arithmetic operations.
\rodinrefsec{Definition}
Addition, subtraction and multiplication behave as expected.
We just describe division and modulo in a more formal way:\\
$a\div b=c \equiv \exists r \qdot r\in 0\upto b \land  b\cdot c + r = a$\\
TODO: The same for modulo.
\rodinrefsec{Types}
\begin{tabular}{l@{\,}l@{\qquad}l@{\,}l@{\qquad}l@{ }l}
$a+b$ & $\in\intg$ & $a\cdot b$ & $\in\intg$ & with & $a \in\intg$\\
$a-b$ & $\in\intg$ & $a\div b$  & $\in\intg$ &      & $b \in\intg$\\
$-a$  & $\in\intg$  & $a\mod b$ & $\in\intg$ \\
\end{tabular}
\rodinrefsec{Well-Definedness}
TODO: Notation? $\wdl$ is actually implemented in Rodin. Is $\wdd$ still interesting?
Or should we use something like the domain condition $\mathcal{DOM}$
because it's simpler for operators?\\
$\wdl(a\div b) \equiv \wdl(a) \land \wdl(b) \land b\neq 0$ \\
$\mathcal{DOM}(\div)(a,b) \equiv b\neq 0$



%%% Local Variables: 
%%% mode: latex
%%% TeX-master: "rodin-doc"
%%% End: 
