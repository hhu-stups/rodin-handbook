\section{Mathematical Notation}
\label{reference_03}

Here we cover the complete mathematical notation of Event-B. For each expression (like an operator) we describe its purpose, its type, the type of its arguments and well-definedness-conditions. We roughly separate the expressions into three groups: predicates, set-theoretical and arithmetic.

Some laws (like the commutative law of addition in arithmetic) would be nice but will not be part of the first iteration of the documentation. E.g. the Z reference manual does this.

References to related proof rules would be nice to have, too. But again, this is not part of the first iteration.

\subsection{Introduction}

What data types exist, what are well-definedness-conditions, how the description of the expressions is organized.

\subsection{Predicates}

All operators that work with predicates ($\land$, $\lor$, quantifier, \ldots). 

\subsection{Sets and relations}

\subsubsection{Sets of functions}
\begin{rrnames}
  $\pfun$  & \texttt{+->}  & Partial functions\\
  $\tfun$  & \texttt{-->}  & Total functions \\
  $\pinj$  & \texttt{>+>}  & Partial injections\\
  $\tinj$  & \texttt{>->}  & Total injections \\
  $\psur$  & \texttt{+->>}  & Partial surjections\\
  $\tsur$  & \texttt{-->>}  & Total surjections \\
  $\tbij$  & \texttt{>->>}  & Bijections \\
\end{rrnames}
\begin{rodinrefentry}
  \rrdesc
  A partial function from $S$ to $T$ is a relation that maps an element of $S$ to at most one element
  of $T$. A function is total if its domain contains all elements of $S$, i.e. it maps every element
  of $S$ to an element of $T$.

  A function is injective (is an injection) if two distinct elements of $S$ are always mapped to distinct
  elements of $T$. An equivalent statement is that the inverse of the function is a also a function.

  A function is surjective (is a surjection) if for every element of $T$ there exists an element in $S$
  that is mapped to it.

  A function is bijective (is a bijection) if it is both injective and surjective.
  \rrdef
  $S \pfun T = \{~ f ~|~ f\in S\rel T \land (\forall e,x,y \qdot e\mapsto x\in f \land e\mapsto y\in f \limp x=y) ~\}$\\
  $S \tfun T = \{~ f ~|~ f\in S\pfun T \land \dom(f) = S ~\}$\\
  $S \pinj T = \{~ f ~|~ f\in S\pfun T \land f^{-1} \in  T\pfun S ~\}$\\
  $S \tinj T = (S\pinj T) \binter (S\tfun T)$\\
  $S \psur T = \{~ f ~|~ f\in S\pfun T \land \ran(f) = T ~\}$\\
  $S \tsur T = (S\psur T) \binter (S\tfun T)$\\
  $S \tbij T = (S\tinj T) \binter (S\tsur T)$\\
  \rrtypes
  With $S\in\pow(\alpha)$, $T\in\pow(\beta)$ and for each operator $\opelipse$ of $\pfun$, $\tfun$, $\pinj$, $\tinj$, $\psur$, $\tsur$, $\tbij$:\\
  $S \opelipse T \in \pow(\pow(\alpha\cprod\beta))$
  \rrwd
  For each operator $\opelipse$ of $\pfun$, $\tfun$, $\pinj$, $\tinj$, $\psur$, $\tsur$, $\tbij$:\\
  $\wdl(S\opelipse T) \defi \wdl(S) \land \wdl(T)$
\end{rodinrefentry}


\subsection{Arithmetic}

\subsubsection{Sets of numbers}
\begin{rrnames}
  $\intg$  & \texttt{INT}  & Integers \\
  $\nat$   & \texttt{NAT}  & Natural numbers, starting with 0 \\
  $\nat_1$ & \texttt{NAT1} & Natural numbers, starting with 1 \\
  $\upto$  & \texttt{..}   & Range of numbers
\end{rrnames}
\begin{rodinrefentry}
  \rrdesc
  The set of all integers is denoted by $\intg$. It contains all elements of the type.
  The two subsets $\nat$ and $\nat_1$ contain all elements greater or equal to 0 resp. 1.
  The range of numbers between $a$ and $b$ is denoted by $a \upto b$.
  \rrdef
  $\nat   = \{~ n ~|~ \in\intg\land n\geq 0~\}$\\
  $\nat_1 = \{~ n ~|~ \in\intg\land n\geq 1~\}$\\
  $a\upto b = \{~ n ~|~ \in\intg\land a\leq n \land n\leq b~\}$
  \rrtypes
  $\intg\in\pow(\intg)$ \\
  $\nat\in \pow(\intg)$ \\
  $\nat_1\in\pow(\intg)$ \\
  $a \upto b\in\pow(\intg)$  with  $a\in\intg$ and $b\in\intg$
  \rrwd
  $\wdl(\intg) \defi \btrue$\\
  $\wdl(\nat) \defi \btrue$\\
  $\wdl(\nat_1) \defi \btrue$\\
  $\wdl(a \upto b) \defi \wdl(a) \land \wdl(b)$
\end{rodinrefentry}

\subsubsection{Arithmetic operations}
\begin{rrnames}
  $+$      & \texttt{+}   & Addition \\
  $-$      & \texttt{-}   & Subtraction or unary minus \\
  $\cdot$ & \texttt{*}   & Multiplication \\
  $\div$   & \texttt{/}   & Integer division \\
  $\mod$   & \texttt{mod} & Modulo \\
\end{rrnames}
\begin{rodinrefentry}
  \rrdesc
  These are the usual arithmetic operations.
  \rrdef
    Addition, subtraction and multiplication behave as expected.

    The division is defined in a way that $1\div 2=0$ and $-1\div 2=0$:\\
    $a\div b = \max(\{c ~|~ c\in\nat \land b\cdot c \leq a\})$ for $a\in\nat$ and $b\in\nat$\\
%    $a\div b=c \defi \exists r \qdot 0\leq r \land r<b \land  b\cdot c + r = a$ for $a\in\nat$ and $b\in\nat$\\
    $(-a)\div b = - (a\div b)$\\
    $a\div (-b) = - (a\div b)$

    TODO: The same for modulo.
  \rrtypes
  With $a\in\intg$ and $b\in\intg$: \\
  $a+b\in\intg$\\
  $a-b\in\intg$\\
  $-a\in\intg$\\
  $a\cdot b\in\intg$ \\
  $a\div b\in\intg$ \\
  $a\mod b\in\intg$
  \rrwd
  $\wdl(a+b) \defi \wdl(a) \land \wdl(b)$ \\
  $\wdl(a-b) \defi \wdl(a) \land \wdl(b)$ \\
  $\wdl(-a) \defi \wdl(a)$ \\
  $\wdl(a\cdot b) \defi \wdl(a) \land \wdl(b)$ \\
  $\wdl(a\div b) \defi \wdl(a) \land \wdl(b) \land b\neq 0$ \\
  $\wdl(a\mod b) \defi \wdl(a) \land \wdl(b) \land b>0$ \\
  TODO: Notation? $\wdl$ is actually implemented in Rodin. Is $\wdd$ still interesting?
  Or should we use something like the domain condition $\mathcal{DOM}$
  because it's simpler for operators?\\
\end{rodinrefentry}

%%% Local Variables: 
%%% mode: latex
%%% TeX-master: "rodin-doc"
%%% End: 
