% defining a if(plastex) environment
\newif\ifplastex
\plastexfalse

\def\doculist#1#2{
\begin{quote}
\hspace{-10mm}
\textrm{\includegraphics[width=7mm]{#2}} % Hack!  We "mark" the image with textrm so that we can use a different CSS-Style in plastex.
\vspace{-8mm}

#1
\end{quote}
}

\def\tick#1{\doculist{#1}{img/tick_64.png}}
\def\info#1{\doculist{#1}{img/info_64.png}}
\def\warning#1{\doculist{#1}{img/warning_64.png}}
\def\pencil#1{\doculist{#1}{img/pencil_64.png}}

% different method to write an ASCII backslash for plastex and normal pdflatex
\ifplastex
  \newcommand{\mybackslash}{\textbackslash}
\else
  % we do not use textbackslash for latex, because it does not use the current font setting
  \newcommand{\mybackslash}{\symbol{`\\}}
\fi

% Path to resources like zip's with machines
% We use a relative path in the html + eclipe version (in order to work offline)
% and an absolute path in the pdf version
\ifplastex
	\newcommand{\filepath}{files/}
\else
	\newcommand{\filepath}{\handbookpath/\versionpath/files/}
\fi

% Use this definition to create a link to the file. The definition takes to arguments. 
% The first argument (1) defines the file name i.e. Celebrity.zip or in case if you saved 
% the file in a subdirectory subdirecotry/Celebrity.zip. The second argument (2) defines 
% the name which should be displayed in the document, i.e. Celebrity Problem Example Download
\def\file#1#2{
\href{\filepath#1}{#2}
}
