% defining a if(plastex) environment
\newif\ifplastex
\plastexfalse

\def\doculist#1#2{
\begin{quote}
\hspace{-10mm}
\textrm{\includegraphics[width=7mm]{#2}} % Hack!  We "mark" the image with textrm so that we can use a different CSS-Style in plastex.
\vspace{-8mm}

#1
\end{quote}
}

\def\tick#1{\doculist{#1}{img/tick_64.png}}
\def\info#1{\doculist{#1}{img/info_64.png}}
\def\warning#1{\doculist{#1}{img/warning_64.png}}
\def\pencil#1{\doculist{#1}{img/pencil_64.png}}

% macro for icons
\def\icon#1{
\includegraphics[]{img/icons/#1}
}
\def\inlineicon#1{
\includegraphics[height=2ex]{img/icons/#1}
}

% macro for image versions (pdf version + html version)
% #1 Path to image for pdf version
% #2 Path to image for html version
% #3 Caption
% #4 Label
\def\imagedpi#1#2#3#4#5{
	\ifplastex
		\begin{figure}[!h]
		\begin{center}
			\includegraphics{#3}
			\caption{#4}
			\label{#5}
		\end{center}
		\end{figure}
	\else
		\begin{figure}[!h]
		\begin{center}
			\includegraphics[width=#2]{#1}
			\caption{#4}
			\label{#5}
		\end{center}
		\end{figure}
	\fi
}

% different method to write an ASCII backslash for plastex and normal pdflatex
\ifplastex
  \newcommand{\mybackslash}{\textbackslash}
\else
  % we do not use textbackslash for latex, because it does not use the current font setting
  \newcommand{\mybackslash}{\symbol{`\\}}
\fi

% Path to resources like zip's with machines
% We use a relative path in the html + eclipe version (in order to work offline)
% and an absolute path in the pdf version
\ifplastex
	\newcommand{\filepath}{files/}
\else
	\newcommand{\filepath}{\handbookpath/\versionpath/files/}
\fi

% Use this definition to create a link to the file. The definition takes to arguments. 
% The first argument (1) defines the file name i.e. Celebrity.zip or in case if you saved 
% the file in a subdirectory subdirecotry/Celebrity.zip. The second argument (2) defines 
% the name which should be displayed in the document, i.e. Celebrity Problem Example Download
\def\file#1#2{
\href{\filepath#1}{#2}
}

% We want to mark contributions from other plugins in a special way, by including the plugin's
% icon and by putting the content in a gray box.  We have to approach this differently for
% Latex and for Plastex:
% Latex: We use "shaded" from package "framed"
% Platexte: We use "verse" as the marker and create the shading with the style sheet.
\newcommand{\tmpName}{Dummy}
\ifplastex
\newenvironment{rodin-plugin}[2]
{
\renewcommand{\tmpName}{#2}
  \begin{verse}
\begin{wrapfigure}{l}{}
    \includegraphics{#1}
\end{wrapfigure}
}
{
\newline
\textit{This contribution requires the \textbf{\tmpName} plugin.  The content is maintained by the plugin contributors and may be out of date.}
\end{verse}
}
\else
\usepackage{framed}
\definecolor{shadecolor}{rgb}{0.93,0.93,0.93}
\newenvironment{rodin-plugin}[2]
{
\renewcommand{\tmpName}{#2} % Otherwise we cannot use #2 in the end block - stupid!
\begin{shaded}
\begin{wrapfigure}{l}{10mm}
\vspace{-5mm}
\includegraphics[width=10mm]{#1}
\vspace{-5mm}
\end{wrapfigure}
\noindent
}
{
\vspace{1mm}
\noindent\rule{\textwidth}{.1pt}
\vspace{1mm}
\noindent
{\scriptsize This contribution requires the \textbf{\tmpName} plugin.  The content is maintained by the plugin contributors and may be out of date.}

\end{shaded}
}
\fi

% Marginpars are  cropped - this formats them nicely.
\let\oldmarginpar\marginpar
\renewcommand\marginpar[1]{\-\oldmarginpar[\raggedleft\scriptsize{#1}]
{\raggedright\small{#1}}}
\marginparwidth=2cm

% A command to typeset names of an Event-B section (like variables, invariant, etc)
% consistently.
\newcommand{\eventbsection}[1]{\textsl{#1}}

% A command to typeset consistently the names of proof obligations
\newcommand{\eventbpo}[1]{\textsf{#1}}

% Event-B's finite operator
\newcommand{\bfinite}{\mathrm{finite}}
\newcommand{\bpartition}{\mathrm{partition}}
\newcommand{\bunaryunion}{\mathrm{union}}
\newcommand{\bunaryinter}{\mathrm{inter}}

% Commands for the structure of the reference section

% rrnames is used for the array of operator symbols and description
% at the beginning of a reference section.
% The environment defines an array with three columns:
% 1) The mathematical symbol
% 2) The ASCII representation
% 3) A description of the operator
\newenvironment{rrnames}%
  {\begin{tabular}{l@{\quad---\quad}l@{\quad---\quad}l}}%
    {\end{tabular}}

% The environment rodinrefentry is used for a reference section with several
% entries: Description, Definition, Types, Well-Definedness
% \rrindent is the indention in such an environment
\newlength{\rrindent}
\setlength{\rrindent}{8em}
\newenvironment{rodinrefentry}{%
   \renewcommand\descriptionlabel[1]{\makebox[\rrindent][r]{\textbf{##1}}}
   \setlength{\leftmargini}{\rrindent}
   \begin{description}%
}{%
   \end{description}%
}
\newcommand{\rrdesc}{\item[Description]}
\newcommand{\rrdef}{\item[Definition]}
\newcommand{\rrtypes}{\item[Types]}
%\newcommand{\rrwd}{\item[Well-Definedness]}
\newcommand{\rrwd}{\item[WD]}
\newcommand{\rrfis}{\item[Feasibility]}
\newcommand{\rrex}{\item[Example]}

\newcommand{\actfis}{\mathcal{F}}

% operators (L and D) for well-definendness
\newcommand{\wdl}{\mathcal{L}}
\newcommand{\wdd}{\mathcal{D}}

% a placeholder symbol for operators
\newcommand{\opelipse}{\mathbin{\Box}}

\newcommand{\podef}[4]{%
  \begin{center}
    \setlength{\parindent}{2em}\vspace{0.2em}
    \begin{tabular}{rp{0.5\textwidth}}
      \hline
      & \textbf{#1} \\
      Name       & #2 \\
      Goal       & #3 \\
      Hypotheses & #4 \\
      \hline
    \end{tabular}
  \end{center}
}

% a second approach to proof obligations
\newcommand{\pode}[3]{%
  \begin{center}
    \setlength{\parindent}{2em}\vspace{0.2em}
    \begin{tabular}{rp{0.6\textwidth}}
      \hline
      & \textbf{#1} \\
      Name       & #2 \\
      Goal       & #3 \\
      \hline
    \end{tabular}
  \end{center}
}

%%% Local Variables: 
%%% mode: latex
%%% TeX-master: "rodin-doc"
%%% End: 
