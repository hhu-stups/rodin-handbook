\section{Further Reference}
\label{reference_10}


\subsection{Abstract Machine}
\label{abstract_machine}

\subsection{Abstract Machine Notation}
\label{abstract_machine_notation}

\subsection{Arithmetic}
\label{arithmetic}

\subsection{Atelier B Provers}
\label{atelier_b_provers}

\subsection{Auto Prover}
\label{auto_prover}

\subsection{Axiom}
\label{axiom}

\subsection{B}
\label{b}

A formal modeling notation.  Two dialects are classical B (\ref{classical_b}) and Event-B (\ref{event_b}).

\subsection{Camille}
\label{camille}

Camille is an alternative, text-based editor.  It can be installed through the Eclipse Install mechanism.  More information is available in the Rodin Wiki (\ref{rodin_wiki}).

\subsection{Classical B}
\label{classical_b}

\subsection{Constant}
\label{constant}

\subsection{Context}
\label{context}

\subsection{Data Refinement}
\label{data_refinement}

\subsection{Datatypes}
\label{datatypes}

List of Event-B datatypes

\subsection{Deadlock}
\label{deadlock}

\subsection{Deploy}
\label{deploy}

The Deploy Project funded the Rodin project in part.  More information at \url{http://www.deploy-project.eu/}.

\subsection{Eclipse}
\label{eclipse}

- Eclipse Definition

- Pointers to Web Tutorials, etc.

\subsection{Editor View}
\label{editor_view}


\subsection{Event}
\label{event}

Definition event

\subsection{Event-B}
\label{eventb}

Event-B is a formal method (\ref{formal_method}) for system-level modelling and analysis. Key features of Event-B are the use of set theory (\ref{set_theory}) as a modelling notation, the use of refinement (\ref{refinement}) to represent systems at different abstraction levels and the use of mathematical proof to verify consistency between refinement levels.

\paragraph{See Also:}
\begin{itemize}
\item \url{http://www.event-b.org}
\end{itemize}

\subsection{Event-B Component}
\label{eventb_component}

Machines (\ref{machine}) and Contexts (\ref{context}) are components.

\subsection{Event-B Explorer}
\label{eventb_explorer}

The View showing the Event-B projects and their content.  In the default Event-B perspective, it is the slim browser on the left edge of the Workspace.  If it is missing, make sure that you use the correct perspective.  You can explicitly enable it with \textsf{Windows $\rangle$ Show View... $\rangle$ Event-B Explorer}.

\subsection{First Order Predicate Calculus}
\label{first_order_predicate_calculus}


\subsection{Formal Method}
\label{formal_method}

\subsection{Gluing Invariant}
\label{gluing_invariant}

\subsection{Goal View}
\label{goal_view}

\subsection{Guard}
\label{guard}

\subsection{IDE}
\label{ide}

Integrated Development Environment

\subsection{Initialization}
\label{initialization}

Every machine has a special event \texttt{INITIALIZATION} that will be used to initialize the machine's state.

TODO: Determinism, refinement.

\subsection{Label}
\label{label}

\subsection{Machine}
\label{machine}

\subsection{Mathematical Notation}
\label{mathematical_notation}

\subsection{Menu Bar}
\label{menu_bar}

\subsection{Model Checker}
\label{model_checker}

\subsection{Naming Convention}
\label{naming_convention}

In this subsection we describe a recommended naming convention.  Good naming conventions save time -- and nerves.

\subsection{Non-Deterministic}
\label{non_deterministic}

\subsection{Outline View}
\label{outline_view}

\subsection{P0 Prover}
\label{p0_prover}

\subsection{Partition}
\label{partition}

\subsection{Proof Control View}
\label{proof_control_view}

\subsection{Proof Obligation}
\label{proof_obligation}

\subsection{Proof Tree View}
\label{proof_tree_view}

\subsection{Plugin}
\label{plugin}

\subsection{Refinement}
\label{refinement}

\begin{description}
	\item[Horizontal Refinement]
	\item[Vertical Refinement]
	\item[Data Refinement]
\end{description}

\paragraph*{See also:}
\begin{itemize}
\item Data refinement in the trafficlight tutorial (\ref{tutorial:data_refinement})
\end{itemize}

\subsection{Rodin Wiki}
\label{rodin_wiki}

\url{http://wiki.event-b.org/}

\subsection{Structural Editor}
\label{structural_editor}

\subsection{Symbols View}
\label{symbols_view}

\subsection{Predicate Logic}
\label{predicate_logic}

\subsection{ProB}
\label{prob}

ProB is an animator and model checker for the B-Method. It allows fully automatic animation of many B specifications, and can be used to systematically check a specification for errors. 

ProB is available as a Rodin Plugin.

\paragraph*{See also:}
\begin{itemize}
\item ProB Website: \url{http://www.stups.uni-duesseldorf.de/ProB}
\item In the tutorial: Section~\ref{tut:prob}
\end{itemize}


\subsection{Project}
\label{project}

\subsection{Proof Obligation Labels}
\label{po_labels}

\subsection{Propositional Calculus}
\label{propositional_calculus}

\subsection{Relation}
\label{relation}

\subsection{Rodin}
\label{rodin}

... Rodin Definition ...

\subsection{Rodin Download}
\label{rodin_download}

Download Page: http://sourceforge.net/projects/rodin-b-sharp/files/Core\_Rodin\_Platform/


\subsection{Rodin Platform}
\label{rodin_platform}



\subsection{Rodin Problems View}
\label{rodin_problems_view}


\subsection{Rodin Nature}
\label{rodin_nature}

Eclipse Projects can have one or more natures to describe their purpose.  The GUI can then adapt to their nature.  Rodin Projects must have the Rodin-Nature.  If you create an Event-B project, it automatically has the right nature.  If you want to modify an existing project, you can edit the \texttt{.project} file and add the following XML in the \texttt{<natures>} section:

\pencil{
\texttt{<nature>org.rodinp.core.rodinnature</nature>}
}

\subsection{Sees}
\label{sees}

\subsection{Set}
\label{set}

\subsection{Set Theory}
\label{set_theory}

... Set Theory Definition ...

\subsection{Temporal Logic}
\label{temporal_logic}

\subsection{Theorem}
\label{theorem}

\subsection{Tool Bar}
\label{tool_bar}

\subsection{Undischarged Proof Obligations}
\label{undischarged_proof_obligations}

\subsection{Upgrade}
\label{Upgrade}


\subsection{Witness}
\label{witness}

\subsection{Wizard}
\label{wizard}

\subsection{Zip File}
\label{zip_file}


